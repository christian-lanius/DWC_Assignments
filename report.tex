%%%%%%%%%%%%%%%%%%%%%%%%%%%%%%%%%%%%%%%%%
% Short Sectioned Assignment
% LaTeX Template
% Version 1.0 (5/5/12)
%
% This template has been downloaded from:
% http://www.LaTeXTemplates.com
%
% Original author:
% Frits Wenneker (http://www.howtotex.com)
%
% License:
% CC BY-NC-SA 3.0 (http://creativecommons.org/licenses/by-nc-sa/3.0/)
%
%%%%%%%%%%%%%%%%%%%%%%%%%%%%%%%%%%%%%%%%%

%----------------------------------------------------------------------------------------
%	PACKAGES AND OTHER DOCUMENT CONFIGURATIONS
%----------------------------------------------------------------------------------------

\documentclass[paper=a4, fontsize=11pt]{scrartcl} % A4 paper and 11pt font size

\usepackage[T1]{fontenc} % Use 8-bit encoding that has 256 glyphs
%\usepackage{fourier} % Use the Adobe Utopia font for the document - comment this line to return to the LaTeX default
\usepackage[english]{babel} % English language/hyphenation
\usepackage{amsmath,amsfonts,amsthm} % Math packages
%\usepackage{enumitem}
\usepackage{bm}
\usepackage{graphicx}
\usepackage{float}
%\usepackage{lipsum} % Used for inserting dummy 'Lorem ipsum' text into the template

\usepackage{sectsty} % Allows customizing section commands
\allsectionsfont{\centering \normalfont\scshape} % Make all sections centered, the default font and small caps

\usepackage{fancyhdr} % Custom headers and footers
\pagestyle{fancyplain} % Makes all pages in the document conform to the custom headers and footers
\fancyhead{} % No page header - if you want one, create it in the same way as the footers below
\fancyfoot[L]{} % Empty left footer
\fancyfoot[C]{} % Empty center footer
\fancyfoot[R]{\thepage} % Page numbering for right footer
\renewcommand{\headrulewidth}{0pt} % Remove header underlines
\renewcommand{\footrulewidth}{0pt} % Remove footer underlines
\setlength{\headheight}{13.6pt} % Customize the height of the header

%\numberwithin{equation}{section} % Number equations within sections (i.e. 1.1, 1.2, 2.1, 2.2 instead of 1, 2, 3, 4)
%\numberwithin{figure}{section} % Number figures within sections (i.e. 1.1, 1.2, 2.1, 2.2 instead of 1, 2, 3, 4)
%\numberwithin{table}{section} % Number tables within sections (i.e. 1.1, 1.2, 2.1, 2.2 instead of 1, 2, 3, 4)

\setlength\parindent{0pt} % Removes all indentation from paragraphs - comment this line for an assignment with lots of text
% packages for code
\usepackage{listings}
\usepackage{color}

\definecolor{dkgreen}{rgb}{0,0.6,0}
\definecolor{gray}{rgb}{0.5,0.5,0.5}
\definecolor{mauve}{rgb}{0.58,0,0.82}

\lstset{frame=tb,
	language=matlab,
	aboveskip=3mm,
	belowskip=3mm,
	showstringspaces=false,
	columns=flexible,
	basicstyle={\small\ttfamily},
	numbers=none,
	numberstyle=\tiny\color{gray},
	keywordstyle=\color{blue},
	commentstyle=\color{dkgreen},
	stringstyle=\color{mauve},
	breaklines=true,
	breakatwhitespace=true,
	tabsize=3
}

\usepackage{makecell}
\usepackage{boldline}
\setcellgapes{3pt}
\usepackage{multirow}
\usepackage[list=true, font=large, labelfont=bf, 
labelformat=brace, position=top]{subcaption}
%----------------------------------------------------------------------------------------
%	TITLE SECTION
%----------------------------------------------------------------------------------------

\newcommand{\horrule}[1]{\rule{\linewidth}{#1}} % Create horizontal rule command with 1 argument of height

\title{	
\normalfont \normalsize 
\textsc{Digital Wireless Communications} \\ [25pt] % Your university, school and/or department name(s)
\horrule{0.5pt} \\[0.4cm] % Thin top horizontal rule
\huge Assignment 1 \\ % The assignment title
\horrule{2pt} \\[0.5cm] % Thick bottom horizontal rule
}

\author{Christian Lanius, 81723448} % Your name

\date{\normalsize May 17, 2018} % Today's date or a custom date

\begin{document}

\maketitle % Print the title

%----------------------------------------------------------------------------------------
%	PROBLEM 1
%----------------------------------------------------------------------------------------

\subsection*{Bit Error Rate for AWGN and Rayleigh Fading Channel}
For this assignment, I implemented a simple simulation of digital communication over a noisy channel. The MATLAB files are attached to this report. The BER vs. $E_B/N_0$ graphs are calculated when the script \textit{assignment.m} script is executed. Two plots are generated which are shown in figure \ref{fig:graph_ber}. Although the plots looks exactly the same, they are obtained by simulating a BPSK and QPSK signal respectively. Because the bit error rate for both is identical, the plots look the same. \par
An additional plot, shown in figure \ref{fig:graph_scatter} is generated showing some scatter plots of the received signal. The signal is equalized with the known impulse response of the channel, otherwise the fading channel's scatter plot would be completely unreadable. The $E_B/N_0$ is as annotated for each plot. One can easily see the symbols slowly seperating for higher SNR, especially for the AWGN. As expected, the scatter plot looks a lot noisier for the fading channel compared to the AWGN.\par
One downside of the code written is that it is implicitly assumed that the same number of symbols is transmitted for each $E_B/N_0$ value. This means that the execution time becomes very long if the error rate should be accurately measured for high SNR. Approximately $10^9$ bits are necessary to expect 10 errors for an $E_B/N_0$ of 12dB for the AWGN channel.\par
\begin{figure}
	\begin{subfigure}[c]{0.55\textwidth}
		
		\includegraphics[width=\textwidth]{graphs/ber_bpsk_cropped.png}
		\subcaption{BPSK}
		
	\end{subfigure}
	\begin{subfigure}[c]{0.55\textwidth}
		\includegraphics[width=\textwidth]{graphs/ber_qpsk_cropped.png}
		\subcaption{QPSK}
	\end{subfigure}
	\caption{The bit error rate versus the energy per bit for BPSK and QPSK for an AWGN and a fading channel. The black symbols are the simulated values, the lines correspond to the theoretical performance.}
	\label{fig:graph_ber}
\end{figure}

\begin{figure}
	\includegraphics[width=1.1\textwidth]{graphs/scatter_plots.png}
	\caption{Scatter plots for different channel, modulation and $E_B/N_0$ combinations.}
	\label{fig:graph_scatter}
\end{figure}
\end{document}